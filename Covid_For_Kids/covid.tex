%%%%%%%%%%%%%%%%%%%%%%%
% PACKAGES NECESSÁRIOS
%%%%%%%%%%%%%%%%%%%%%%%
\documentclass[12pt,brazil,a4paper]{article}
\usepackage[portuguese]{babel}
\usepackage[utf8]{inputenc}
\usepackage{tcolorbox}


%%%%%%%%%
% ARTIGO
%%%%%%%%%
\begin{document}
\title{\textbf{Covid for Kids}}
\author{Alexya Oliveira, Gabriel Fanto Stundner, Lucas Leal, Matheus Santos \\ \\ Pontifícia universidade Católica do Rio Grande do Sul \\ Escola Politécnica \\ Curso de Engenharia de Software}
\maketitle

\textbf{Design de Interação 2020/1}

\textit{Prof. Sílvia Moraes}


\section{Sumário}
\begin{description}
Gerar Indice Aqui
\end{description}
\section{Contexto e Objetivo}
\begin{description}
Propostas de Interface
\end{description}
\section{Escopo e Requisitos}
\begin{description}
Elabore uma breve descrição do projeto e seu escopo. Descreva a abordagem que será usada (gamificação, quiz, atividades, ...). Se o projeto ainda não tiver nome, batize-o. 
\end{description}
\subsection{Requisitos Funcionais}
\begin{tcolorbox}
\textbf{Nome: } Exibir de forma lúdica e pedagógica informações úteis sobre o vírus e prevenção \\
\textbf{Descrição: } \\
\tab\tab \textbf{Como: } Criança    \\
\tab\tab \textbf{Eu quero: } Jogar um jogo de luta contra o Corona vírus\\
\tab\tab \textbf{Para que: } Eu possa aprender formas de prevenção na quarentena\\
\textbf{Notas: } \\
\textit{a) } O jogo tem que ser simples para crianças\\
\textit{b) } O jogo tem que ser didático   
\end{tcolorbox}
\subsection{Requisitos Não-Funcionais}
\section{Modelos de Usuário}
\begin{description}
Quem são os usuários (atuais e futuros) ?” 
\end{description}
\subsection{Perfil de Usuários}
\subsection{Personas}
\section{Modelos de Trabalho}
\begin{description}
\subsection{Personas}

\end{description}
\section{Prototipação}
\begin{description}
Organizar em subseções para descrever os itens do enunciado.
\end{description}
\section{Avaliação}
\begin{description}
Descrição da avaliação do protótipo. Realizar a avaliação junto aos usuários e documentar o feedback. Indicar ainda as melhorias feitas no protótipo que foram decorrentes da avaliação dos usuários e do stakeholder.
\end{description}
\section{Considerações Finais}
\begin{description}
Comentar o trabalho, no que ele contribuiu para sua aprendizagem e as dificuldades encontradas.
\end{description}




\end{document}